\section{Introduction}
\label{sec.introduction}

\cappos{I talk about hypervisors too here, in part because I feel they are a
natural solution to the same problem.  I probably shouldn't because that makes
our scope seem larger than what we can support w/ our results.}
\yiwen{I think it is a good idea to mention hypervisors, and we can explicitly point out 
the scope of our work, which is focused only on the OS kernels, not the hypervisors.}

Privileged code \cappos{such as OS kernels and hypervisors} is an essential 
component of modern computer systems but 
presents a number of security challenges. The privileged code itself is vulnerable 
to attacks. When vulnerabilities in this code are exploited, other parts 
of the system could be damaged.
%\yanyan{why separate these two para? TCB is also privileged code.}
For example, failures in the trusted computing base (TCB) can cause detrimental crashes, 
such as complete system failures, privilege escalation, etc. 
Decreasing the feasibility of exploitation of the kernel bugs, especially privilege escalation, 
would be a substantial step toward stronger security for computer systems.

Bugs in operating system kernels have motivated development of a diverse set of
technologies to reduce these risks, such as OS virtualization, 
system call filtering, library OSes, etc. \cappos{However, there are two
major problems with these systems.  First,}  Unfortunately, these technologies 
also harbor vulnerabilities that are exploitable.  \cappos{Second,}
Even with these technologies in place, applications from the user space 
could still have access to a portion of the kernel that might contain 
bugs and is risky to be exposed. \yanyan{Better use examples in these
two cases to back up the argument.}
\yiwen{Right, I will look for proper examples.}

One contributing factor to these security problems 
%associated with these existing technologies and potential
%new designs 
is that it is still unknown which portions of the privileged code are
safe to expose, and which portions would be vulnerable and exploitable. 
One key missing puzzle is a standard method for quantifying the safety (or 
risk) of the privileged code. 
For example, is it a good practice to minimize \textit{the number of lines of code}?
Is \textit{the number of API calls} a good metric for security?  \cappos{cites
for these would be good.} \yiwen{will add later}
Is avoiding files that have historically had bugs a viable approach?
\yanyan{Instead of asking questions, using citations
will be stronger, eg, xxx used loc, yyy used number of calls. Questions feel abrupt..}
\cappos{Need to refer to the sorts of sources mentioned in this paper: 
``Does Bug Prediction Support Human Developers? Findings From a Google 
Case Study".  This refers to fairly close related work in the SE
community.  If mentioned here, it should be a sentence clause and a few
citations.  Section 3 can say more.}
To the best of our knowledge, there is no quantitative metric \cappos{that
is considered an effective way} to evaluate the 
security of privileged code \cappos{perhaps end the sentence here?  I feel
the rest gets away from what we want to say.} 
%and provide insights into how to design a secure system that 
%only interacts with the privileged code in a safe manner.

In this paper, we propose a metric that quantitatively measures and evaluated 
the security of the Linux kernel version 3.13.0-generic. \cappos{security of the Linux kernel versions X-Y} 
\yiwen{Linux kernel version added} 
%The kernel trace profiling data is an important piece of information 
%obtained by using our metric. 
In our metric, we obtained the kernel trace data about which lines of kernel code 
get executed by running user applications  
\cappos{I don't understand the previous sentence.
I think it may be clearer if rephrased.}
\yiwen{rephrased} 
This data reveals the use pattern of the kernel code 
based upon applications running from the user space. The use pattern can then be
leveraged to design new systems that have to meet specific requirements, 
such as strong security or high performance. 
\yanyan{I don't think it's appropriate to say "use pattern" can be leveraged to design
something. Maybe the result of our data analysis?}
To design and build secure systems, 
our key hypothesis is that commonly used kernel paths contain fewer bugs, and are 
safe to be exposed. 
\yanyan{I feel this sentence (hypothesis) should come earlier, maybe 
after the first sentence in this para.}
Results from using our metric indicate that this hypothesis is reasonable, and 
can be leveraged to create new design aiming at building secure systems. 

To protect the kernel from being exploited, we came up with a new design that 
we call ``safely-reimplement'', 
\cappos{do we want to name it here?  We should probably call it 
`safely-reimplement' so we can refer to it later.}
which aims at providing strong isolation between the kernel space and the user space, 
and thus providing better protection to the kernel. Our new design leverages 
our metric to make better design decisions and build more effective system. 
We implemented a sandbox system, Lind, based on this new design.
Lind minimizes the amount of risky privileged code that is reachable by a
sandboxed program.  To allow a program that uses risky functionality to
execute correctly, risky functionality is reimplemented inside a sandbox with
a small trusted computing base (TCB) \cappos{comprised of safe code}. 
This additional level of sandboxing provides an outlet for risky functionality, without which
legacy programs will not run, while containing security flaws in this code. 

We evaluate Lind by comparing it against other similar systems 
that are not designed using our metric. 
\yanyan{will this raise an issue about cherry picking?}
\yiwen{this is a really good question and a real concern, we may need some explanation about 
how we pick the competing systems.}
More specifically, we run user applications
under Lind and other systems, and compare their kernel traces. In addition, we examined historical
kernel bug reports to verify whose kernel trace is likely to trigger more
kernel bugs. \cappos{I'm confused.  Do we use fuzzing or real apps for
this?}
\yiwen{for the comparison evaluation, I think it is more realistic to run user apps and some packages.} 
Evaluation results showed that running applications in Lind is the least likely to trigger kernel bugs, 
which indicates that designs, such as the prototype of Lind that uses our metric, 
tend to result in more secure systems. 

The main contributions of this paper are as follows:

\begin{itemize}
\item We propose a novel metric for quantitatively measuring and evaluating 
the security of privileged code, such as in a hypervisor or kernel. 
Our metric examines the kernel trace generated by running popular user 
applications and produces recommendations at the lines-of-code level.  

\item Using our metric, we have findings that substantiated our key hypothesis that commonly used kernel paths 
contain fewer bugs.  \cappos{I wonder if this should be phrased differently.
Of course these results are things we found to be true.  If not, why would
we write the paper?  Do you want to say that we validated this on Linux
over certain versions of the kernel?}
\yiwen{I need to think about this.}

\item We created a novel secure design paradigm ``safely-reimplement'' that 
came from examining and leveraging our metric and key hypothesis 
that commonly used kernel paths contain fewer bugs.  \cappos{Probably need a
sentence to describe it here.} 
\yiwen{added} 
Our paradigm only uses safe functionality and reimplement risky functionality inside a sandbox to contain potential risks. 
Using this new design, we implemented a sandbox we called Lind, which provides more secure environment
for running applications and provide strong protection to the kernel. 

\item Our evaluation results showed that implementation of our sandbox Lind did not trigger any of the 
zero-day vulnerabilities we examined, 
while systems built without using our metric have chances to trigger
\cappos{What does this mean?  It sounds like we don't know.} 
vulnerabilities, which suggests 
that our metric can help design and build more secure systems effectively.
\yanyan{This sentence is too long.}
\end{itemize}

The remainder of this paper is organized as follows. 
We discuss the motivation that drives our work in \S{2}. 
In \S{3}, we propose our metric as the solution to solve the problem and provide detailed discussion about the metric.
New architecture we designed using our metric is introduced in \S{4}. In \S{5}, we discuss the implementation of the design, a sandbox called Lind. 
Evaluation results of Lind and the new design strategy are presented in \S{6}. 
We then discuss related work in \S{7} and conclude in \S{8}. 
