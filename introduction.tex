\section{Introduction}
\label{sec.introduction}

\cappos{I talk about hypervisors too here, in part because I feel they are a
natural solution to the same problem.  I probably shouldn't because that makes
our scope seem larger than what we can support w/ our results.}
\yiwen{I think it is a good idea to mention hypervisors, and we can explicitly point out 
the scope of our work, which is focused only on the OS kernels, not the hypervisors.}

Privileged code such as OS kernels and hypervisors is an essential 
component of modern computer systems but 
presents a number of security challenges. The privileged code itself is vulnerable 
to attacks. When vulnerabilities in this code are exploited, other parts 
of the system could be damaged.
%\yanyan{why separate these two para? TCB is also privileged code.}
For example, failures in the trusted computing base (TCB) can cause detrimental crashes, 
such as complete system failures, privilege escalation, etc. \yanyan{these 
examples are still too abstract. Please use more concrete examples.}
Decreasing the feasibility of exploitation of the kernel bugs, especially privilege escalation, 
would be a substantial step toward stronger security for computer systems.

Bugs in operating system kernels have motivated development of a diverse set of
technologies to reduce these risks, such as OS virtualization, 
system call filtering, library OSes, etc. However, there are two
major problems with these systems.  First, these technologies 
also harbor vulnerabilities that are exploitable.  Second,
even with these technologies in place, applications from the user space 
could still have access to a portion of the kernel that might contain 
bugs and is risky to be exposed. \yanyan{Better use examples in these
two cases to back up the argument.}
%\yiwen{Right, I will look for proper examples.}

One contributing factor to these security problems 
%associated with these existing technologies and potential
%new designs 
is that it is still unknown which portions of the privileged code are
safe to expose, and which portions would be vulnerable. 
One key missing puzzle is a standard method for quantifying the safety (or 
risk) of the privileged code. 
For example, is it a good practice to minimize \textit{the number of lines of code}?
Is \textit{the number of API calls} a good metric for security?  \cappos{cites
for these would be good.} %\yiwen{will add later}
Is avoiding files that have historically had bugs a viable approach~\cite{lewis2013does}?
\yanyan{Instead of asking questions, using citations
will be stronger, eg, xxx used loc, yyy used number of calls. Questions feel abrupt..}
\cappos{Need to refer to the sorts of sources mentioned in this paper: 
``Does Bug Prediction Support Human Developers? Findings From a Google 
Case Study".  This refers to fairly close related work in the SE
community.  If mentioned here, it should be a sentence clause and a few
citations.  Section 3 can say more.}
\yanyan{I added this citation here~\cite{lewis2013does}. Is this the right place?}
To the best of our knowledge, there is no quantitative metric that
is considered an effective way to evaluate the 
security of privileged code. 
%\cappos{perhaps end the sentence here?  I feel
%the rest gets away from what we want to say.} 
%and provide insights into how to design a secure system that 
%only interacts with the privileged code in a safe manner.

System calls are the interface between user space and the kernel. 
%The kernel trace profiling data is an important piece of information 
%obtained by using our metric. 
To quantitatively measure and evaluate the security of privileged 
code, Linux kernel version 3.13.0-generic in our case, we 
obtain the kernel trace data about which lines of kernel code 
get executed by invoking system calls through user applications.  
%\cappos{I don't understand the previous sentence.
%I think it may be clearer if rephrased.}
%\yiwen{rephrased} 
Our key hypothesis is that \textit{commonly used kernel paths 
contain fewer bugs, and are safe to be exposed}. 
Our analysis of this data reveals the use pattern of the kernel code, 
and confirms our hypothesis (\S{\ref{sec.metric}}). Therefore, we use 
\textbf{kernel coverage} as the metric to guide the design of new 
systems that provide strong security (\S{\ref{sec.design}}, 
\S{\ref{sec.implementation}}). 
%\yanyan{I don't think it's appropriate to say "use pattern" can be leveraged to design
%something. Maybe the result of our data analysis?}
%To design and build secure systems, 
%\yanyan{I feel this sentence (hypothesis) should come earlier, maybe 
%after the first sentence in this para.}
The results indicate that %this hypothesis is reasonable, and 
the kernel coverage metric can be leveraged to build secure systems
(\S{\ref{sec.evaluation}}). 

To protect the kernel from being exploited, we aim to minimize the 
kernel coverage of a system. Therefore, we ``safely-reimplement'' 
%\cappos{do we want to name it here?  We should probably call it 
%`safely-reimplement' so we can refer to it later.}
system calls that are exposed to applications. 
This provides strong isolation between the kernel space and the user space, 
and better protection to the kernel. %Our new design leverages 
%our metric to make better design decisions and build more effective system. 
We implement a sandbox system, \textbf{Lind}.
Lind minimizes the amount of risky privileged code that is reachable by a
sandboxed program.  To allow a program to %that uses risky functionality to
execute correctly, risky system calls are reimplemented inside a sandbox with
a small trusted computing base (TCB) comprised of safe code. 
This additional level of sandboxing provides an outlet for risky functionality, without which
legacy programs will not run, while containing security flaws in this code. 

We evaluate Lind by comparing it against other similar systems.
More specifically, we run user applications
under Lind and other systems, and compare their kernel coverage. In addition, we examined historical
kernel bug reports to verify which kernel trace is more likely to trigger 
kernel bugs. 
Evaluation results showed that running applications in Lind is the least likely to trigger kernel bugs. 
This indicates that minimizing kernel coverage tends to result in more secure systems. 

The main contributions of this paper are as follows:

\begin{itemize}
\item We propose a novel metric for quantitatively measuring and evaluating 
the security of privileged code, such as in a hypervisor or kernel. 
Our metric examines the kernel trace generated by running popular user 
applications and produces recommendations at the lines-of-code level.  

\item Using our metric, we have findings that substantiated our key hypothesis that commonly used kernel paths 
contain fewer bugs.  \cappos{I wonder if this should be phrased differently.
Of course these results are things we found to be true.  If not, why would
we write the paper?  Do you want to say that we validated this on Linux
over certain versions of the kernel?}
\yiwen{I need to think about this.}

\item We created a novel secure design ``safely-reimplement'' 
by examining and leveraging our metric and key hypothesis 
that commonly used kernel paths contain fewer bugs.  
Our design uses safe system calls and reimplements risky system 
calls inside a sandbox to contain potential risks. With this new design, 
we implement a sandbox Lind that provides secure environment
for applications and strong protection to the kernel. 

\item Our evaluation results show that the implementation of Lind does 
not trigger any of the zero-day vulnerabilities we examined, 
while systems built without using our metric have chances to trigger
\cappos{What does this mean?  It sounds like we don't know.} 
vulnerabilities. This suggests 
that our metric can help design and build more secure systems effectively.
\end{itemize}

The remainder of this paper is organized as follows. 
We discuss the motivation that drives our work in \S{\ref{sec.motivation}}. 
In \S{\ref{sec.metric}}, we propose our kernel coverage metric to solve the existing 
security problems. %and provide detailed discussion about the metric.
A new architecture using this metric is introduced in \S{\ref{sec.design}}. 
In \S{\ref{sec.implementation}}, we describe the implementation of our design, 
the sandbox Lind. Evaluation results of Lind and the new design strategy 
are presented in \S{\ref{sec.evaluation}}. We then discuss related work in 
\S{\ref{sec.related_work}} and conclude in \S{\ref{sec.conclusion}}. 
