\section{Related Work}
\label{sec.related_work}

Many previous ideas and systems have helped and inspired us. We categorize and discuss closely related work below.

\cappos{Given the organization of the paper, I'm not sure if other software 
engineering techniques to find bugs fit in here too.  For example, the FixCache
work and follow on papers.  I would be tempted to put them in
the early sections instead, but I haven't really thought about it much.}

\subsection{Measuring and Profiling the Kernel}

To our best knowledge, our proposed metric is the first quantitative metric to profile the kernel trace. 
There are other work that tried to measure and profile the kernel. However, their goals and approaches 
had substantial differences from ours.
\yiwen{I need to add related work in this area and briefly discuss the differences
between their approaches and ours.} 


\subsection{Historical Kernel Bug Analysis}
Historical kernel bug reports indicate the vulnerabilities hidden in the kernel space. 
Previous work has studied many of the historical kernel bugs and gathered statistics. 
However, there has yet to be any concrete method to match any specific kernel bug to the kernel code.  
Our metric could fill in the gap and leverage kernel bug reports to help new system designs more effectively.
\yiwen{More related work needs to be added.}

\subsection{Isolation Between the Kernel Space and the User Space}

As an instantiation of new designs using our metric, our sandbox system Lind aims at building a highly secure system 
that provides strong isolation between the kernel space and the user space. Many existing techniques also tried to 
address security issues by isolating the kernel space and the user space. In this section, we look closely into the techniques 
that share similar security goals as Lind. We further categorize those techniques into three categories and discuss them 
in \S{7.3.1}, \S{7.3.2}, and \S{7.3.3}. 

\subsubsection{Virtualization}

Programming language virtualization, such as Java, Silverlight, JavaScript and Flash are commonly used sandboxes that 
have achieved widespread adoption. These sandboxes combine untrusted application code with an interpreter and 
standard libraries. Standard libraries consolidate routines to perform I/O, network communication, and other system sensitive 
functionality. Though many sandboxes implement the bulk of their standard libraries in a memory-safe language like 
Java or C\#, flaws in memory-safe code can still pose a threat.

Virtualization is widely used to isolate untrusted applications that share the same hardware. There is bare metal hardware 
virtualization, such as VMware ESX Server, XEN \cite{Xen:03}, and MS Hyper-V. There is also hosted hardware virtualization, 
such as VMware Workstation, VMware Server, MS Virtual PC and Server. However, there are limitations with virtualization. 
Virtualization hypervisor suffers from attack vectors including architectural vulnerability, software vulnerability, 
and configuration risk. For example, small code footprint of hypervisor could become a potential vulnerability 
and lead to serious problem.

Drawbridge \cite{Drawbridge:11} uses light-weight processes and a library OS to present a Windows persona to 
a wide variety of Windows applications. This is accomplished by moving a large portion of the OS into the process, 
and presenting a simplified system virtual machine like interface to each process. This approach brings many of 
the benefits of VM based temporal, spatial and fault isolation properties to a per-process level.


\subsubsection{System Call Interposition}

System call interposition offers a number of properties that make it attractive for building sandboxes, 
though it can be error prone \cite{SCI:04}. Approaches for delegation and filtering have been extensively studied, 
along with their respective tradeoffs with security and performance. Janus Version 2 (J2) \cite{Janus0:96, Janus:99} 
uses filtering and sandboxing, while Ostia \cite{SCI:04} uses a delegation.

Ostia also provides a hybrid interposition architecture, which allows for kernel level enforcement and user policies. 
The Kernel module enforces policy, denies direct access to restricted resources, and delegates certain calls to 
the Emulation library, which sends transformed system calls to the Agents. The Agent reads the policy file and 
handles the delegation of calls.


\subsubsection{Software Fault Isolation}

Software Fault Isolation (SFI) is a sandboxing technique in which native instructions are executed, but only those instructions 
which can be executed without violating the sandbox's constraints are allowed \cite{SFI:93}. Instructions usually excludes 
are those which could jump outside the program's predefined memory region or execute a system call. 
Programs must use a specific compiler which enforces these constraints, then the programs are validated on load. 
Nooks \cite{Nooks:03} isolates Linux kernel modules, so that when they crash the whole kernel does not crash as well. 
They use two techniques: sandboxing and micro-reboots. Nooks uses both as SFI-based and user-level sandbox. 
Micros-reboots conceptually refresh small parts of the system.
