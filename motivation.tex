\section{Motivation}
\label{sec.motivation}

The Operating System kernel is a critical component of the entire computing system. 
However, current OS kernels are not protected properly and is vulnerable under the 
current security model of existing computing systems. In this section, we argue that 
it is crutial to have precise information about which portions of the OS kernel are safe, 
and have better control over the interface exposed by the kernel.

\subsection{Kernel protection deficiencies}
The kernel has to provide an interface to serve the requests from applications for acquiring 
essential system resources, such as memory, I/O, and CPU. 
All applications running in the system depend on the kernel to provide 
essential functionality.  Thus all code, even that is 
sandboxed or untrusted, will make calls that are eventually processed by the
kernel.
%
However, the OS kernel is privileged code that needs to be well protected. Access to the
kernel code should be avoided whenever possible and be conducted in a careful manner. 
%whenever contact is necessary. 
Failing to do so would lead to potential exploitation of kernel bugs and 
may lead to serious damage to the entire computing system.  

%\cappos{There is a shift in focus here.  We need to signal this somehow.}
However, not all kernels provide sufficient protection against exploitation.
These kernels have rendered themselves the target of attackers. Over the past few years, 
the exploitation of OS kernels has been wide spead, 
despite the amount of efforts by researchers and practitioners. 
There were 125 reported vulnerabilities of the Linux kernel, and 215 reported vulnerabilities 
of all types of kernels in 2014~\cite{NVD:14}. In contrast, the exploitation 
of user-level software has become much harder. For example, highly 
targeted applications, such as browsers and document viewers, have 
started to employ sandboxing technique. Compare to user-level 
applications, the kernel has a huge codebase and an attack surface 
that keeps increasing due to the constant addition of 
new features \cite{Metrics:13}. As an example, the size of the Linux kernel in terms of lines of code 
has more than doubled, from 6.6 MLOC in v2.6.11 to 16.9 MLOC in v3.10 \cite{Linux:13}.

Such a huge kernel codebase has lead to excessive kernel exploitation. As a result, the Linux kernel 
has been plagued by common software flaws, such as stack and heap buffer overflows 
\cappos{It seems like overkill to have all CVEs have their own cite.  Perhaps
just have one cite for the CVE system and then just list the number?}
\yiwen{will fix this} 
\cite{CVE:20093234, CVE:20131828, CVE:20132892}, NULL pointer and pointer arithmetic errors 
\cite{CVE:20050736, CVE:20092698}, memory disclosure vulnerabilities 
\cite{CVE:20093002, CVE:20104073}, use-after-free and format string bugs 
\cite{CVE:20132852, CVE:20134343}, signedness errors \cite{CVE:20103437, CVE:20132094}, 
integer overflows \cite{CVE:20050736, CVE:20102959}, race conditions 
\cite{CVE:20091527, CVE:20093547}, as well as missing authorization checks and 
poor argument sanitization vulnerabilities 
\cite{CVE:20103904, CVE:20104347, CVE:20120946, CVE:20130268}. 
%\yanyan{Listing all the names of bugs is unnecessary, 2-3 is enough.}
The exploitation of these bugs is mostly 
%despite the existence of kernel protection mechanisms, 
due to the weak separation between user and kernel space.
%\cappos{What point are you trying to make with the last bit here?}
%\yiwen{I was trying to say that exploitation of bugs is effective because of bad separation 
%between user space and the kernel.}

\subsection{Design of secure systems}
To build a secure computing system, it is critical to provide sufficient protection for the kernel and 
prevent kernel exploitation. Many previous attempts, including OS virtualization, 
system call filtering, library OSes, etc., have tried to address this problem. However, 
kernel exploitation still exist, and the problem has not been solved 
effectively.
\cappos{You also would be making a stronger point if you could support the claim
by showing flaws in those types of systems.} 
%which leads to a fundamental question, why is it not working and what are we missing? 
%\yanyan{The sentences above read familiar? (compared to what have been said in the 
%previous two paragraphs)}
%After thinking about this question and 

Previous work and our experience show that OS designers and developers 
are not clear about which part of the kernel are safe, and thus can be 
exposed without potential dangers. 
%\yanyan{It will feel much stronger of we say "Our experience shows that OS developers (or some 
%other related people), do not know clearly...."}
Many previous work, such as building a sandbox and using library OSes, 
focuses on isolating user programs. While isolation 
can be useful to mitigate the problem, the proposed systems   
access the kernel the same way as before. 
%because they do not know what's the best way to leverage the kernel securely. Essentially, 
Compared to no isolation, these approaches move
the attack surface from between the user space and the kernel, to 
between these new systems and the kernel. The attack surface
is not necessarily reduced.  
%\cappos{Is `shrinking' really the right thing
%to consider?  I think we will argue against LOC as the metric.} 
%\yanyan{The attack surface is not necessarily reduced? Note: spell out doesn't as does not.}
As a result, the proposed solutions do not make the systems more secure. 
%
It would be more effective if we can reduce the attack 
surface and have better control over the interface exposed by the 
kernel. In other words, we need to have precise information about 
which portions of the kernel are safe and which portions are risky.
 
%\cappos{The text that follows feels like bridge text into the next section.
%If so, it is much too long.  If you're trying to say what potential value
%a metric would have, then you probably need to start a new paragraph and
%make it clearer.  However, I think you talk about the value of the metric
%in 3.1 as well, so you will likely want to combine text to avoid
%duplication.}
To obtain this information, we propose a novel security metric to 
quantitatively measure and evaluate the kernel: kernel coverage at 
the level of \textit{lines of code}.
%Our metric will provide insights into the kernel and let us 
%understand security features of the kernel in a better way. 
%
With this metric, we can understand security features of the kernel, 
and control the interface exposed to the user space in a secure way. 
With such control, we propose a design of secure systems that provides 
strong protection to the kernel. Ideally, we would like to only expose 
the safe portion of the kernel to the user space. However, in some 
cases, we have to expose certain risky portion. With the 
kernel coverage metric, we know clearly about the potential dangers 
and safely-reimplement those risky kernel interface (or system calls) 
to mitigate the potential damage. 
%\yanyan{Without saying what this metric is, this paragraph reads a bit moot.}