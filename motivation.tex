\section{Motivation}
\label{sec.motivation}

The Operating System (OS) kernel is a critical component of the entire computing system, 
yet is not protected properly and very vulnerable under the current security model of existing 
computing systems.

The kernel has to provide an interface to serve the requests from user applications for acquiring 
essential system resources, such as memory, I/O, CPU, and etc. 
All applications running in the system depend on the kernel to provide 
essential functionality.   \cappos{Thus all code, even that which is 
sandboxed or untrusted, will make calls that are eventually processed by the
kernel.}

However, the OS kernel is privileged code that needs to be well protected. Access to the
kernel code should be avoided as much as possible and be conducted in a very careful manner 
whenever contact is necessary. Failing to do this would lead to potential exploitation of kernel bugs and 
may lead to serious damage to the entire computing system.  

\cappos{There is a shift in focus here.  We need to signal this somehow.}
The OS kernel has always been an appealing target. During the past few years, 
the exploitation of OS kernels has become a very serious problem that remains to be solved, 
in spite of large amount of efforts devoted by researchers and practitioners. 
There were 125 reported vulnerabilities of the Linux kernel, and 215 reported vulnerabilities 
of all kinds of kernels in 2014 \cite{NVD:14}. Admittedly, the exploitation of user-level software 
has become much harder, as recent versions of popular OSes come with numerous protections 
and exploit mitigations. The principle of least privilege is better enforced in user accounts 
and system services, compilers offer more protections against common software flaws, 
and highly targeted applications, such as browsers and document viewers, have started to 
employ sandboxing technique. On the other hand, the kernel has a huge codebase and 
an attack surface that keeps increasing due to the constant addition of 
new features \cite{Metrics:13}. Indicatively, the size of the Linux kernel in terms of lines of code 
has more than doubled, from 6.6 MLOC in v2.6.11 to 16.9 MLOC in v3.10 \cite{Linux:13}.

Opportunities for kernel exploitation are abundant. As an example consider the Linux kernel, 
which has been plagued by common software flaws, such as stack and heap buffer overflows 
\cappos{It seems like overkill to have all CVEs have their own cite.  Perhaps
just have one cite for the CVE system and then just list the number?}
\cite{CVE:20093234, CVE:20131828, CVE:20132892}, NULL pointer and pointer arithmetic errors 
\cite{CVE:20050736, CVE:20092698}, memory disclosure vulnerabilities 
\cite{CVE:20093002, CVE:20104073}, use-after-free and format string bugs 
\cite{CVE:20132852, CVE:20134343}, signedness errors \cite{CVE:20103437, CVE:20132094}, 
integer overflows \cite{CVE:20050736, CVE:20102959}, race conditions 
\cite{CVE:20091527, CVE:20093547}, as well as missing authorization checks and 
poor argument sanitization vulnerabilities 
\cite{CVE:20103904, CVE:20104347, CVE:20120946, CVE:20130268}. 
The exploitation of these bugs is particularly effective, 
despite the existence of kernel protection mechanisms, 
due to the weak separation between user and kernel space.
\cappos{What point are you trying to make with the last bit here?}

To build a secure computing system, it is critical to provide sufficient protection for the kernel and 
prevent kernel exploitation from happening. Many previous attempts, including OS virtualization, 
system call filtering, library OSes and etc. have tried to address this problem. However, 
kernel exploitation still exist, and the problem has not been solved 
effectively \cappos{This is a long sentence.  You may want to break it up.  }
\cappos{You also would be making a stronger point if you could support the claim
by showing flaws in those types of systems.}, which leads to a 
fundamental question, why is it not working and what are we missing? After thinking about this
question and examining previous work, we find that one key missing puzzle is that people do not know 
clearly which part of the kernel are safe and can be exposed without potential dangers. Many previous
work, such as building a sandbox and using library OSes focused on isolating user programs. While isolation 
can be useful and help mitigate the problem, the proposed systems still get access to the kernel. 
And the key is: the way they access the kernel has no difference than before, because they do not 
know what's the best way to leverage the kernel securely. So, essentially, what they are doing is just moving 
the attack surface from the previous place (between the user space and the kernel) to another place 
(between the new proposed system and the kernel). The attack surface
doesn't necessarily shrink.  \cappos{Is `shrinking' really the right thing
to consider?  I think we will argue against LOC as the metric.}
Therefore, the proposed solutions were not effective. 

It would be more effective, if we can actually shrink the attack surface and have better control over 
the interface exposed by the kernel. In another word, we need to be crystal clear about which portions of 
the kernel are safe and which portions of the kernel are risky.
\cappos{The previous sentence is a key point.  We need to draw it out and
justify it more.}  
\cappos{The text that follows feels like bridge text into the next section.
If so, it is much too long.  If you're trying to say what potential value
a metric would have, then you probably need to start a new paragraph and
make it clearer.  However, I think you talk about the value of the metric
in 3.1 as well, so you will likely want to combine text to avoid
duplication.}
To obtain this piece of critical information, we
propose a novel security metric to quantitatively measure and evaluate the kernel. Our metric will provide
insights into the kernel and let us understand security features of the kernel in a better way. 

With the help from our metric, we would be able to process the power of controlling the interface 
exposed to the user space in a precise and secure way. We can then design and build secure systems 
that provide stronger protection to the kernel. Ideally, we would like to only expose safe portion 
of the kernel to the user space. Realistically, in some cases, even if we have to expose certain risky portion 
of the kernel, we know clearly about the potential dangers and therefore can come up with better 
strategies to mitigate the potential damage. 
