\section{Conclusion}
\label{sec.conclusion}

Isolating untrusted user applications from the underlying kernel is desirable, in order to protect the privileged code and 
avoid the exploitation of bugs. However, there has yet to be a standard method that can shed lights on how to effectively
isolate the user space applications from the kernel space without losing desired functionality.

We propose a new metric that quantitatively measures and evaluates the kernel code being executed when running
user applications. We have a key hypothesis that commonly used kernel paths contain few bugs. Using our metric, 
we have findings suggesting the hypothesis is reasonable and solid. 

A new design for building secure systems was created based on the findings from our metric. 
We implemented a sandbox system, called Lind, which is based upon the new design using our metric. 
Our system Lind, securely reconstruct complex yet essential OS functionality inside a sandbox. 
The sandbox itself is designed to have a minimized trusted computing base (TCB) and only interact with the kernel
in a minimal and safe way. 

Our evaluation results have shown that Lind is least likely to trigger historically reported kernel bugs, compared against
other systems built without using our metric, such as VirtualBox and Graphene. The successful implementation of Lind
suggests that new designs using our metric are likely to lead to more secure systems. 