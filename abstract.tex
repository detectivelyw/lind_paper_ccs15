\begin{abstract}

The Operating System (OS) kernel is an essential component of modern computer systems, 
yet despite substantial effort, kernels still contain bugs and are exploitable. 
In this paper, we explained why existing techniques failed to protect the kernel effectively. 
Moreover, to provide a solution, we devised a new metric that quantitatively measures and 
evaluates the lines of code in the kernel that get executed to determine which portions of 
the kernel are safe. Our key hypothesis is that commonly used kernel paths contain few bugs.   
Results from using our metric suggest that our hypothesis is reasonable. 

A new design for building secure systems was created using our metric. We implemented 
a prototype system Lind, using our new design, which accesses only commonly used kernel paths, 
and have a very small trusted computing base that places only limited trust in the privileged code. 
Our experimental results show that Lind triggered less kernel bugs than other systems not using our 
design. Therefore, the new design using our metric tends to result in more secure systems.  

\end{abstract}